In this paper we have formalized the concept of Micro City, defining its aspects and characteristics that determine its distinctiveness. 

We identified the amusement park as a possible scenario in which this concept could be applied and analyzed the case-study-specific characteristics.

At a later time, we conducted research about amusement parks in general and what are the technologies they currently use to improve the visitors' experience.
Then, we chose Mirabilandia as a real-world amusement park case study, conducting an analysis of the current technological status made available to the park and then focusing on what cutting-edge technologies could implement our Micro City concept.
In particular, the state of the art about situated recommendation techniques and visitor location exploiting Wi-Fi networks to increase location accuracy were studied. 

Finally, some considerations were made about possible choices for deploying a working system.

This project gave us the opportunity to engage in identifying, developing and partly formalizing an entirely new model. 

It allowed us to study the literature on areas we had never explored before and then adapt the notions learned to our Micro City context.

It was challenging to adapt the theorized concepts into a real-world case study, but we feel satisfied with the results obtained despite the short time available.

\section{Technologies}

\subsection{Wi-Fi as outdoor localization system}

Wi-Fi is a family of wireless network protocols, which are commonly used for local area networking of devices and Internet access, allowing nearby digital device to exchange data by radio waves.

To date, Wi-Fi is a technology that has been pervasively adopted and made accessible to anyone because of cost, ease of installation and configuration.

Unlike other technologies such as the global positioning system (GPS), Wi-Fi was not designed to perform device and/or person location. However, by exploiting ad-hoc techniques it is possible
to leverage this tool to perform device localization. Finally, GPS does not always perform well in any context: in closed environments or where the GPS signal cannot reach, localization using
this technology is approximate or even impractical.
Another problem with GPS is its high power consumption, which is a serious challenge to battery-based mobile devices.
To tackle the problems with GPS, many researchers have proposed a series of alternative localization schemes, including cellular-based systems \cite{ibrahim2010cellsense},
infrared-based systems, ultrasonic-based systems, and radio frequency (RF)-based systems~\cite{bahl2000radar, youssef2002probabilistic}.

Many researchers have proposed a variety of different schemes for outdoor localization based on mobile
devices. These schemes can be divided into two groups: range-based and range-free methods.

\begin{itemize}
      \item \textbf{Range-based}: range-based methods are mainly based on relative distance, which can be obtained through measuring methods like time-of-arrival (ToA),
            time difference of arrival (TDoA), or propagation model generated from RSSI value.
      \item \textbf{Range-free}: one of the most widely used range-free method is fingerprint localization method. This method can be categorized into three types:
            visual fingerprint-based localization, motion fingerprint-based systems, and signal fingerprint-based methods. In our context the latter is the most promising.
\end{itemize}

\subsubsection{Signal fingerprint-based localization}
Signal fingerprint-based localization is widely used in places where a large number of Wi-Fi infrastructures are deployed. This methods commonly consist of offline training phase
and online fingerprint matching phase. The goal of the first phase is to form a fingerprint database which stores the correlation between \textit{Received Signal Strength} (RSS) from
various \textit{Access Points}(APs) and fix locations. The device's location is determined at the matching stage. In this process we use matching algorithm to search the fingerprint
in the database which has the minimum difference with the device that needs to be located. The associated label is our estimated location.

In terms of effect, signal fingerprint-based localization has the ability to get fine-grained results. However, it is impractical to transfer this method directly from the indoor to the
outdoor environment, since there are many problems such as complicated training work and complex outdoor conditions.

% Il resto della sezione la sposterei in una sezione successiva in cui identifichiamo la tecnologia migliore per il tracciamento delle persone. 
% A new approach in signal fingerprint-based localization is proposed in~\cite{du2018hybrid}. In the article they propose the following technique to stimate the position
% of a device: we use $f = {r_1, r_2, \ldots, r_n}$ to indicate a \textit{fingerprint} where $r_i$ represents the RSS value of captured AP and $n$ the number of APs in the fingerprint.
% We calculate the dissimilarity between two fingerprints based on RSS difference. Denote with $\sigma_i = | r_i - r^{'}_{i}|$ the difference of fingerprints $f^{'}$ and $f$ at each $A_i$ where
% $A_i \in A$ where $A$ is the fingerprint's APs set. Due to the fact that two fingerprints may contain different set of APs, so it may appear that AP $A_i$ appears in $f$ but
% does not appear in $f^{'}$. For this situation, we assume the signal strength is weak and let the missing value equal to $-100$. The dissimilarity between $f$ and $f^{'}$ can be shown as:

% \begin{equation}
%     \eta(f, f^{'}) = \sqrt{\sum_{i=0}^{p} \sigma_i^{2}}
% \end{equation}

% where $p = | A \cup A^{'} |$

% We found the sample with minimum dissimilarity through compare all samples stored in the fingerprint database $F$ with the query fingerprint $f$.

% \begin{equation}
%     f^{*} = \arg \max_{f_i \in F} \ \eta(f, f_i)
% \end{equation}

% $L(f^{*})$ is the corresponding location of $f^{*}$ which represent the estimated place.

\subsection{Bluetooth as outdoor localization system}
Bluetooth is a data transmission standard for personal wireless networks. It provides a standard way to exchange information between
devices through a short-range frequency capable of detecting devices covered by the radio signal within about ten meters by putting them
in communication with each other.

Despite being a technology conceived more than 20 years ago, it boasts massive adoption in many contexts such as medical, industrial, and in
recent years in the Internet of Things (IoT). This standard was designed with the goal of achieving low power consumption, short range
and a low cost of production. Over the years, there have been several updates to the protocol, aimed at improving on the one hand the efficiency of
communication by enabling higher transmission frequencies and on the other hand to improve the energy efficiency of devices.

Bluetooth was not designed to perform localization functionality. However, it is possible to exploit certain
features of the protocol to make a more or less precise estimate of a device's location. In particular, the technique most
used to achieve this is that based on the signal strength identified by the RSSI (Receives Signal Strength Indicator).
Other techniques, such as fingerprint-base localization can be employed to estimate position.

With the RSSI-based technique, the RSSIs of all reachable Bluetooth devices are initially acquired, and through techniques of trilateration, the
location of the device is estimated.\\
Fingerprint-based techniques estimate the location by operating in two stages: the first is the training phase that deals with building fingerprints,
which is a record that associates a location with the RSSIs of beacons reachable from that
point. The second phase involves identifying the fingerprint that least deviates from the position where the device is at a given time, and the
position associated to that fingerprint determines the estimated position. Several projects and applications rely on these two techniques to estimate
the position of a device~\cite{mcconville2021vesta, samuel2021smart}.

The proliferation of location services for various IoT applications needs to detect device locations with very high accuracies, on the order of
centimeters. With the introduction of the Bluetooth 5.1 standard, a new feature called \textit{Direction Finding} that enables pinpoint localization
of Bluetooth devices.
This new localization feature provides two different options for positioning a Bluetooth device, namely Angle of Arrival (AoA) and Angle of Departure
(AoD) compared to the previous version that relies only on the received signal strength indication to localize a Bluetooth device.
To be more precise, the former technique allows a receiver equipped with a multi-antenna array to identify the angular position of a transmitter
based on the phase delay of the signal received from the transmitter; the latter allows the transmitting device with multiple antennas to transmit a
radio signal that permits the receiver to determine the directional angle to the transmitter

\section{Technologies}

\subsection{Wi-Fi as outdoor localization system}

Wi-Fi is a family of wireless network protocols, which are commonly used for local area networking of devices and Internet access, allowing nearby digital device to exchange data by radio waves.

To date, Wi-Fi is a technology that has been pervasively adopted and made accessible to anyone because of cost, ease of installation and configuration.

Unlike other technologies such as the global positioning system (GPS), Wi-Fi was not designed to perform device and/or person location. However, by exploiting ad-hoc techniques it is possible
to leverage this tool to perform device localization. Finally, GPS does not always perform well in any context: in closed environments or where the GPS signal cannot reach, localization using
this technology is approximate or even impractical.
Another problem with GPS is its high power consumption, which is a serious challenge to battery-based mobile devices.
To tackle the problems with GPS, many researchers have proposed a series of alternative localization schemes, including cellular-based systems \cite{ibrahim2010cellsense},
infrared-based systems, ultrasonic-based systems, and radio frequency (RF)-based systems~\cite{bahl2000radar, youssef2002probabilistic}.

Many researchers have proposed a variety of different schemes for outdoor localization based on mobile
devices. These schemes can be divided into two groups: range-based and range-free methods.

\begin{itemize}
      \item \textbf{Range-based}: range-based methods are mainly based on relative distance, which can be obtained through measuring methods like time-of-arrival (ToA),
            time difference of arrival (TDoA), or propagation model generated from RSSI value.
      \item \textbf{Range-free}: one of the most widely used range-free method is fingerprint localization method. This method can be categorized into three types:
            visual fingerprint-based localization, motion fingerprint-based systems, and signal fingerprint-based methods. In our context the latter is the most promising.
\end{itemize}

\subsubsection{Signal fingerprint-based localization}
Signal fingerprint-based localization is widely used in places where a large number of Wi-Fi infrastructures are deployed. This methods commonly consist of offline training phase
and online fingerprint matching phase. The goal of the first phase is to form a fingerprint database which stores the correlation between \textit{Received Signal Strength} (RSS) from
various \textit{Access Points}(APs) and fix locations. The device's location is determined at the matching stage. In this process we use matching algorithm to search the fingerprint
in the database which has the minimum difference with the device that needs to be located. The associated label is our estimated location.

In terms of effect, signal fingerprint-based localization has the ability to get fine-grained results. However, it is impractical to transfer this method directly from the indoor to the
outdoor environment, since there are many problems such as complicated training work and complex outdoor conditions.

% Il resto della sezione la sposterei in una sezione successiva in cui identifichiamo la tecnologia migliore per il tracciamento delle persone. 
% A new approach in signal fingerprint-based localization is proposed in~\cite{du2018hybrid}. In the article they propose the following technique to stimate the position
% of a device: we use $f = {r_1, r_2, \ldots, r_n}$ to indicate a \textit{fingerprint} where $r_i$ represents the RSS value of captured AP and $n$ the number of APs in the fingerprint.
% We calculate the dissimilarity between two fingerprints based on RSS difference. Denote with $\sigma_i = | r_i - r^{'}_{i}|$ the difference of fingerprints $f^{'}$ and $f$ at each $A_i$ where
% $A_i \in A$ where $A$ is the fingerprint's APs set. Due to the fact that two fingerprints may contain different set of APs, so it may appear that AP $A_i$ appears in $f$ but
% does not appear in $f^{'}$. For this situation, we assume the signal strength is weak and let the missing value equal to $-100$. The dissimilarity between $f$ and $f^{'}$ can be shown as:

% \begin{equation}
%     \eta(f, f^{'}) = \sqrt{\sum_{i=0}^{p} \sigma_i^{2}}
% \end{equation}

% where $p = | A \cup A^{'} |$

% We found the sample with minimum dissimilarity through compare all samples stored in the fingerprint database $F$ with the query fingerprint $f$.

% \begin{equation}
%     f^{*} = \arg \max_{f_i \in F} \ \eta(f, f_i)
% \end{equation}

% $L(f^{*})$ is the corresponding location of $f^{*}$ which represent the estimated place.

\subsection{Bluetooth as outdoor localization system}
Il bluetooth e' uno standard per la trasmissione di dati per reti personali wireless. Fornisce un modo standard per scambiare informazioni tra
dispositivi attraverso una frequenza a corto raggio in grado di rilevare i dispositivi coperti dal segnale radio entro una decina di metri mettendoli
in comunicazione tra loro.

Nonostante sia una tecnologia pensata piu' di venti anni fa, vanta una massiva adozioni in molti contesti come quello medico, industriale e negli
ultimi anni nell'Internet of Things (IoT). Questo standard venne progettato con l'obiettivo di ottenere bassi consumi, un corto raggio d'azione
e un basso costo di produzione. Negli anni si sono susseguiti diversi aggiornamenti al protocollo, volti a migliorare da un lato l'efficienza di
comunicazione, abilitando frequenze di trasmissione piu' elevate e dall'altro lato migliorare l'efficienza energetica dei dispositivi.

Anche in questo caso il Bluetooth non e' stato pensato per svolgere funzionalità di localizzazione. Tuttavia, e' possibile sfruttare alcune
caratteristiche del protocollo per effettuare una stime piu' o meno precisa della posizione di un dispositivo. In particolare la tecnica maggiormente
utilizzata per raggiungere questo obiettivo e' quella basata sulla potenza del segnale identificata dall'RSSI (Receives Signal Strength Indicator).
Altre tecniche, come il fingerprint-base localization possono essere impiegate per stimare la posizione.

Con la tecnica basata su RSSI, si acquisiscono inizialmente gli RSSI di tutti i dispositivi Bluetooth raggiungibili e mediante tecniche di
trilaterazione si stima la posizione del dispositivo. Le tecniche basate su fingerprint stimano la posizione operando in due fasi: la prima e' la fase
di training che costruisce i fingerprint, ovvero un record che associa una posizione agli RSSI dei beacon raggiungibili da quel punto. La seconda fase
prevede di identificare il fingerprint che meno di discosta dalla posizione in cui il dispositivo si trova in un dato momento e la posizione associata
a quel fingerprint determina la stima della posizione. Diversi progetti e applicazioni si basano su queste due tecniche per stimare la posizione di un
dispositivo~\cite{mcconville2021vesta, samuel2021smart}.

La proliferazione di servizi di localizzazione per diverse applicazioni IoT necessita di rilevare le posizioni dei dispositivi con accuratezze molto
elevate, nell'ordine dei centimetri. Con l'introduzione dello standard Bluetooth 5.1 e' stata introdotta una nuova funzionalità chiamata
\textit{Direction Finding} che abilita una localizzazione puntuale dei dispositivi Bluetooth. This new localization feature provides
two different options for positioning a Bluetooth device, namely Angle of Arrival (AoA) and Angle of Departure (AoD) compared to the previous version
that relies only on the received signal strength indication to localize a Bluetooth device.

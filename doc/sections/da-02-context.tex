\section{Context}\label{sec:context}

The concept of \textit{MicroCity} was born observing contexts with common characteristics, in which people may benefit from \textbf{self-awareness} and \textbf{situatedness} mechanisms in order to have a better experience.

\textit{Micro Cities} concern areas with bounded temporal and spatial extension and with a high concentration of people called guests.
Inside the \textit{Micro City}, different types of activities may be carried out: these might be services (offering experiences or products), that are always operative and available, or events, that happen at a specific time and have a limited time duration.
Most of the people inside the \textit{Micro City} are significantly interested in the available activities that are operative during the \textit{Micro City}'s lifetime.
The latter may vary depending on the specific context of the \textit{Micro City}, and it defines temporal bounds for activities.
Thus, what we can say about a \textit{Micro City} is:
\begin{itemize}
    \item It can be assumed that all the guests are endowed with a wearable device (like a smartphone), so that they can interact with activities.
    \item It is limited in space: it presents well-defined physical bounds, that specify the domain's limits.
    \item It is limited in time: it presents well-defined operation periods, that specify when guests can benefit from activities.
    \item It presents heterogeneous activities that are physically situated and distributed inside it.
    These activities justify the existence of the \textit{Micro City} because they are the reason why guests go to the \textit{Micro City} in the first place.
    Activities may be static, meaning that they cannot physically move, or dynamic, which means that they may move if necessary.
    It can be assumed that activities can gather and send interesting information to guests.
    Finally, activities are able to satisfy a certain amount of guests with a certain frequency.
    Therefore, the number of guests that can benefit from an activity simultaneously is limited.
    \item The guests that take part in the \textit{Micro City} may be individuals or groups of people, and they change, partially or completely, periodically.
    It can be assumed that guests are highly interested in the activities and that groups of guests have similar interests, so they move together inside the \textit{Micro City}.
    Moreover, one may assume that a group of guests uses a single wearable device in order to benefit from the provided services.
    A \textit{Micro City} also presents internal operators, distinguished from guests, that do not benefit from activities (because they manage them).
    \item The high amount of guests that attend activities may cause the increase of waiting time before benefiting from them.
    This can also result in the formation of queues.
    \item Guests may pay a certain amount of money (fee) in order to access the \textit{Micro City} and/or to benefit from activities.
    \item Activities may be proactively recommended to the guests, depending on their wearable-tracked position and their interests.
    \item If guests accept the recommendations given to them, they may receive rewards.
    Rewards may vary depending on the context of the \textit{Micro City} itself, and they may concern:
        \begin{itemize}
            \item A discount applied on a particular product/service or every activity inside the \textit{Micro City}.
            \item A cashback that may promote sustainable actions and behaviours.
            \item Points that can be accumulated and that allow guests to collect prizes offered by the \textit{Micro City}.
            \item The improvement of an experience, such as the reduction of waiting time in a queue.
        \end{itemize}
    \item A \textit{Micro City} presents a map that may suggest or impose routes to reach the activities inside it.
    Guests can locate activities and orientate using the map.
\end{itemize}
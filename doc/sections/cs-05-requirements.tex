\section{Requirements}\label{sec:requirements}

The \textbf{situated recommendation system} must satisfy the following business requirements in order to be appealing to amusement parks:
\begin{enumerate}
    \item The ability to make recommendations based on certain criteria.
    Thus, it should be able to access information about the park in order to elaborate an as desirable as possible recommendation for the visitors.
    It is not important to focus on the way the recommendation is generated.
    Instead, it should be able to define different strategies and use the most suitable one.
    For instance, some strategies may be based on the length of queues or visitors' interests.
    \item The ability to keep track of the state of every attraction inside the park.
    In such a way, it would be possible to provide not only static recommendations (for instance based on the visitors' preferences) but also dynamic ones.
    The latter may be based on attractions' information such as their current queue, their duration, their capacity, etc.
    \item The ability to keep track of the state of every visitor (or group of visitors) inside the park.
    In this way, it would be possible to provide them with recommendations based on their interests and their physical location.
    Thus, visitors would be more satisfied as they would benefit from attractions they surely enjoy.
    Moreover, it could be possible to suggest attractions near to them in order to minimize the walking time.
    \item The possibility to memorize the information collected during the amusement park's lifetime.
    This allows the parks' managers to analyze trends and visitors' preferences in order to make improvements to their attractions or promote others.
    Moreover, this could help them develop new recommendation strategies that might be more effective.
\end{enumerate}
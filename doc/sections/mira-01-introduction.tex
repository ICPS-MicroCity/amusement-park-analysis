\section*{Introduction}
\addcontentsline{toc}{section}{Introduction}
\label{sec:introduction}

The objective of this chapter is to describe the current Mirabilandia amusement park's situation in terms of technologies used in its business and to offer an overview of other parks that already use some suitable technologies for the Micro City context.
Finally, it will be illustrated a potential architecture for a future Mirabilandia improved with all of the characteristics of a Micro City.

In table~\ref{tab:terms}, is listed the terminology of the Micro City context -- specifically for the amusement parks -- extracted from the case study ubiquitous language, which will be used in the document from now on.

\begin{longtable}{|l|p{.4\textwidth}|}
	\hline
	\textbf{Term} & \textbf{Definition} \\
	\hline
	\ul{Visitor}{A person attending the amusement park.}
	\ul{Group of Visitors}{A set of visitors attending the amusement park.}
	\ul{Attraction}{Type of activity offered to visitors. An attraction is continuously available during the amusement park's lifetime and allows visitors to benefit from it at any time. They can be rides, roller coasters, water slides, but also restaurants or shops.}
	\ul{Show}{Type of activity offered to visitors. A show takes place in a specific moment and is carried out only once; when it terminates, it won't be available anymore.}
	\ul{Wearable}{Device owned by each visitor (or group of visitors) that allows them to interact with the amusement park.}
	\ul{Recommendation}{A proposal to benefit from a specific attraction or show in exchange for a reward.}
	\ul{Recommend}{The action of sending a recommendation to the visitors.}
	\ul{Reward}{The recompense received by the visitors that accept a recommendation.}
	\caption{Terms that will be used in the following chapters.}
	\label{tab:terms}
\end{longtable}